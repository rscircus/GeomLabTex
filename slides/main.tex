\documentclass{beamer}
\usetheme{metropolis}           % Use metropolis theme
\title{On Proportional Symbol Maps - An applied perspective}
\date{\today}
\author{David Gödecke, Philip Mayer, Roland Siegbert}
\institute{Geometry Lab SS 2020}
\begin{document}

  \maketitle

  \begin{frame}{Overview - ToC}
    \tableofcontents
  \end{frame}

  \section{Introduction}

  \begin{frame}{Motivation}
    A few words on why.
    Picture of COVID-19 and similar data.
  \end{frame}

  \begin{frame}{Proportional Symbol Maps}
    Backreference.
    Explanation of topic.
  \end{frame}

  \begin{frame}{Maps and Glyphs}
    Introduce glyph types discussed and create transition into Algo section.
  \end{frame}

  \section{Algorithms}

  \begin{frame}{Prior}
    David's part. See Philip's list and or discussion.
  \end{frame}

  \begin{frame}{Generalized}
    See Philip's list and or discussion.
  \end{frame}

  \begin{frame}{Our approach}
    See Philip's list and or discussion.
  \end{frame}

  \begin{frame}{Squares and Pies... and so on}
    See Philip's list and or discussion.
  \end{frame}

  \section{Experimental results}

  \begin{frame}{Simply paste your stuff here, Philipp}
    Yay, Pics! :-)
  \end{frame}

  \section{Exploration in App}

  \begin{frame}{Exploration of the data}
    [Switch to app and play!]
  \end{frame}

  \section{Conclusion and Outlook}

  \begin{frame}{Summary}

    \begin{itemize}
      \item Four glyphs were shown, with two new approaches.
      \item NP-hardness of new approaches was outlined.
      \item Heuristics and greedy approach usually are good choices.
      \item Square/pie approach can be interpreted as discrete version of the relative visibility.
      \item All of this was verified on the most recent COVID-19 data,
      \item and experimentally demonstrated.
    \end{itemize}
  \end{frame}

\end{document}
